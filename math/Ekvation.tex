\documentclass[12pt]{article}
\usepackage{amsmath}
\usepackage{graphicx}
\usepackage{hyperref}
\usepackage[latin1]{inputenc}

\title{Att lösa [$\int_{3}^{9}\ln({x^2 -9})dx = \int_{0}^{a}\ln({x})dx$]}
\author{Alain Mosnier}
\date{2019/01/30}

\begin{document}
\maketitle

\section {Ekvationen}
\begin{equation}
\int_{3}^{9}\ln({x^2 -9})dx = \int_{0}^{a}\ln({x})dx
\end{equation}
Lös ut a.
\section {Generaliserade integraler}
\item Först konstaterar vi att båda sidor av ekvationen är generaliserade integraler (funktionen $\ln$ är inte definierad för variabelvärdet noll). Vi behöver därför fundera på huruvida det vi försöker lösa faktiskt är en giltig ekvation.
\item Vi har tur. Alla integraler på formen:
\begin{equation}
\int_{0}^{a}\ln({x})dx, a \in {\rm I\!R}_{>0}
\end{equation}
är konvergenta.
För att bevisa det kan man till exempel konstatera att:
\begin{equation}
\int_{0}^{1}\ln({x})dx=\lim_{x \to 0}(x-x\ln x-1)=-1
\end{equation}
och att:
\begin{equation}
\forall a \in {\rm I\!R}_{>0}, \exists c \in {\rm I\!R} / \int_{0}^{a}\ln({x})dx=\int_{0}^{1}\ln({x})dx+c
\end{equation}
($c=\int_{a}^{1}ln({x})dx$ eller $c=\int_{1}^{a}ln({x})dx$, beroende på huruvida $a<1$ eller $a\geq1$)
\section {Lösning}
Efter att med några enkla steg ha konstaterat att:
\begin{equation}
\int_{3}^{9}\ln({x²-9})dx=\int_{0}^{12}\ln({x})dx
\end{equation}
så kan man dra slutsatsen att det bara finns en lösning: $a=12$.
\section {Generalisering?}
Anledningen att vänstertermen ovan går att förenkla på ett passande sätt är att $3+3=3²-3$. Går det att generalisera för fler värden än 3 och 9? Dvs vilka lösningar har ekvationen $2y=y²-y$?
\item Svar: $y=0$ eller $y=3$.
\item Ekvationen går alltså inte att generalisera på ett enkelt sätt.
\end{document}
